\documentclass[20pt]{article}
\usepackage[utf8]{inputenc}
\usepackage[T2A]{fontenc}
\usepackage[english]{babel}
\usepackage[russian]{babel}
\usepackage{amsfonts}
\usepackage{amsmath}
\usepackage{amssymb}
\usepackage{arcs}
\usepackage{fancyhdr}
\usepackage{float}
\usepackage[left=3cm,right=3cm,top=3cm,bottom=3cm]{geometry}
\usepackage{graphicx}
\usepackage{hyperref}
\usepackage{multicol}
\usepackage{stackrel}
\usepackage{xcolor}
\usepackage[fontsize=12pt]{fontsize}
\pagestyle{plain}

\begin{document}
\pagestyle{empty}
\normalsize

\title{Лабораторная работа №1 \\
Выбор темы и формирование требований}
\author{Агафонов Вадим M3310 367787}
\date{}
\maketitle

\centerline{\LARGE\textbf{«Система кредитного отдела банка»}}

\section{Описание возможностей системы}

\textbf{Система "Кредитного отдела банка"} предназначена для оптимизации обработки и аналитики процессов, которые связаны с кредитованием банка. Основные функции системы включают:



\begin{itemize}
    \item \textbf{Обработка заявок на кредит}: Создание, отправка и обработка зявок, поступающих на получение кредита.
    \item \textbf{Скоринг и оценка кредитоспособности}: Автоматизированное анализирование кредитоспособности лица, подавшего заявки.
    \item \textbf{Управление документами}: Предоставление возможностей по созданию, изменению, хранению, пересылке документов.
    \item \textbf{Коммуникация с клиентами}: Возможность выйти на контакт с клиентов по всем возможным каналам связи (чат, SMS, мессендджеры, социальные сети и т.д.).
    \item \textbf{Отчетность и аналитика}: Создание отчетных документов для оценки кредитного портфеля и мониторинга его состаяния.
    \item \textbf{Интеграция с внешними системами}: Взаимодействие с другими кредитными организация, государственными органами и т.д.
\end{itemize}

\section{Основные действующие лица}

\begin{itemize}
    \item \textbf{Клиенты}: Физические и юридические лица, подающие заявку на кредитование.
    \item \textbf{Кредитные специалисты}: Сотрудники, отвечающие обработку и анализ кредитных заявок.
    \item \textbf{Менеджеры по работе с клиентами}: Сотрудники, которые общаются с клиентом, сопровождая их от начала и до конца.
    \item \textbf{Администраторы системы}: Технические специалисты, поддерживающие работоспособность системы, также работающие над повышением её отказоустойчивости.
    \item \textbf{Юристы}: Специалисты, занимающиеся юридическими аспектами всех соглашений и документов. 
    \item \textbf{Специалисты по безопасности}: Специалисты, отвечающие за безопасность система, сохранность данных, предотвращение мошенничества и кибер-атак.
    \item \textbf{Аналитический отдел}: Внешний сервис специалистов, занимающихся финансовым анализом и прогнозированием поведения экономической ситуации, предоставляющих информацию об продуктивности отдела и отвечеющие за отчетность, предоставляющих информацию об продуктивности отдела.
\end{itemize}

\section{Действия, доступы и возможности для выделенных лиц}

\subsection{Клиенты}
\begin{itemize}
    \item \textbf{Действия}: Создание, изменение, отправление заявки, предоставление документов.
    \item \textbf{Доступы}: Личный кабинет пользователя.
    \item \textbf{Возможности}: Мониторинг статуса заявки, общение со специалистами через каналы связи (чат или электронная почта)
\end{itemize}

\subsection{Кредитные специалисты}
\begin{itemize}
    \item \textbf{Действия}: Обработка заявок, принятие решения о предоставление кредита.
    \item \textbf{Доступы}: Система управления заявками, документы клиентов, система скоринга.
    \item \textbf{Возможности}: Просмотр, изменение заявок, составление отчетов и предоставление их руководству.
\end{itemize}

\subsection{Менеджеры по работе с клиентами}
\begin{itemize}
    \item \textbf{Действия}: Общение с клиентами, консультация по кредитным продуктам.
    \item \textbf{Доступы}: Документы клиентов, чат, корпоратиная электронная почта.
    \item \textbf{Возможности}: Создание заявки, редактирование заявки, запрос и предоставление документов.
\end{itemize}

\subsection{Администраторы системы}
\begin{itemize}
    \item \textbf{Действия}: Обслуживание системы, настройка корпоративных компьютеров, настройка и поддержание работы сервера, решение технических проблем, контоль ошибок.
    \item \textbf{Доступы}: Полный доступ к системе и ее настройкам.
    \item \textbf{Возможности}: Настройка системы управление правами пользователей, обеспечение безопасности данных.
\end{itemize}

\subsection{Юристы}
\begin{itemize}
    \item \textbf{Действия}: Создание, проверка и подготовка юридических документов, консультация по юридическим вопросам.
    \item \textbf{Доступы}: База данных документов.
    \item \textbf{Возможности}: Просмотр, создание и редактирование документов.
\end{itemize}

\subsection{Специалисты по безопасности}
\begin{itemize}
    \item \textbf{Действия}: Обеспечение защиты от любых видов атак, мошенничества, мониторинг системы.
    \item \textbf{Доступы}: Полный доступ к системе безопасности и данным.
    \item \textbf{Возможности}: Предотвращение атак, изменение пользовательских прав доступа, контроль за данными.
\end{itemize}

\subsection{Аналитический отдел}
\begin{itemize}
    \item \textbf{Действия}: Анализирование и прогнозирование экономического положения, анализ данных, мониторинг портфеля, мониторинг работы сотрудников, принятие стратегических решений.
    \item \textbf{Доступы}: Доступ к данным, относящимся к финансовой составляющей отдела, Общие данные работы кредитного отдела, инструменты для анализа.
    \item \textbf{Возможности}: Создание, редактирование и предоставление отчетов..
\end{itemize}



\section{Описание ценности приложения}

\begin{itemize}
    \item \textbf{Ускорить обработку заявок} Автоматизированность системы помогает специалистам быстрее анализировать заявку и принимать решение.
    \item \textbf{Повысить точность оценки кредитоспособности}: Повышение безопасности кредитного отдела банка от недобросовестных лиц, действия которых ведут к финансовым потерям.
    \item \textbf{Улучшить взаимодействие с клиентами}: Поддержка клиентов на всех стадиях позволяет улучить обслуживание и доверие клиентов.
    \item \textbf{Обеспечить прозрачность и контроль}: Невозможность фальсификации и мошенничества со стороных сотрудников компании, благодаря контролю за заявкой на всех этапах разными специалистами.
    \item \textbf{Повысить безопасность данных}: Интеграция с внешними системами позволяет повысить безопасность данных и защитить конфиденциальную информацию.
    \item \textbf{Оптимизировать управление персоналом}: Возможность контроля персонала и моментальной коммуникации с ним позвоялет повысить эффективность отдела.
\end{itemize}

\bibliographystyle{alpha}


\end{document}